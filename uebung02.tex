% Systemmagement und Sicherheit
% uebung2

\documentclass[11pt,a4paper,ngerman]{article}
\usepackage[utf8]{inputenc}
\usepackage[left=2.5cm,right=2.5cm,top=1.5cm,bottom=0.5cm,includeheadfoot]{geometry}
\usepackage{fancyhdr}
\usepackage{titlesec}
\pagestyle{fancy}
\usepackage[makeroom]{cancel}
\fancyhead[L]{C.Wolf, K.Schultz} %Kopfzeile links
\fancyhead[C]{Systemmanagement und Sicherheit} %zentrierte Kopfzeile
\fancyhead[R]{\thepage} %Kopfzeile rechts
\cfoot{}
\fancyfoot[R]{}
\date{06.05.2017}
\author{K. Schultz, C. Wolf}
\title{Systemmanagement und Sicherheit Uebung2}

\titleformat{\section}
{\normalfont\Large\bfseries}{\thesection}{1em.}{}

\usepackage{listings}
\usepackage{color}

\definecolor{dkgreen}{rgb}{0,0.6,0}
\definecolor{gray}{rgb}{0.5,0.5,0.5}
\definecolor{mauve}{rgb}{0.58,0,0.82}

\lstset{
  frame=tb,
  language=C,
  aboveskip=3mm,
  belowskip=3mm,
  showstringspaces=false,
  columns=flexible,
  basicstyle={\small\ttfamily},
  numbers=none,
  numberstyle=\tiny\color{gray},
  keywordstyle=\color{blue},
  commentstyle=\color{dkgreen},
  stringstyle=\color{mauve},
  breaklines=true,
  breakatwhitespace=true,
  tabsize=3,
  escapeinside={(*}{*)}
}

\begin{document}


\section{Aufgabe}
\begin{enumerate}
\item rm:
	\begin{enumerate}
		\item löscht Dateien, aber Standardmäßig keine Directories
		\item /bin/rm
		\item rm test.txt
		\item unlink()
	\end{enumerate}
\item mv:
	\begin{enumerate}
		\item verschiebt Datein, kann auch zum umbennen genutzt werden
		\item /bin/mv
		\item mv test.txt ~/Documents/test.txt
		\item access() und rename()
	\end{enumerate}
\item chmod:
	\begin{enumerate}
		\item aendert die Zugriffsrecht von Dateien
		\item /bin/chmod
		\item symbolisch: chmod +rwx file, binaer: chmod 777 file
		\item pathconf()
	\end{enumerate}
\item chown:
	\begin{enumerate}
		\item aendert den Besitzer einer Datei
		\item Debian: /bin/chown, FreeBSD: /usr/bin/chown
		\item chown [user] file
		\item munmap(), exit()
	\end{enumerate}
\item mkdir:
	\begin{enumerate}
		\item erstellt Ordner
		\item /bin/mkdir
		\item mkdir name
		\item sigprocmask()
	\end{enumerate}
\item rmdir:
	\begin{enumerate}
		\item loescht Ordner
		\item /bin/rmdir
		\item rmdir name
		\item execve(), mmap()
	\end{enumerate}
\item kill:
	\begin{enumerate}
		\item sendet ueber die PID Signale an Prozesse
		\item /bin/kill
		\item kill [PID]
		\item lstat()
	\end{enumerate}
\item ln:
	\begin{enumerate}
		\item steht fuer link, erzeugt Verknuepfung zu Datei oder Ordner
		\item /bin/ln
		\item ln \[option\] ZIEL \[name_der_verknuepfung\]
		\item openat(), fstat()
	\end{enumerate}
\item sleep:
	\begin{enumerate}
		\item laesst den Prozess fuer eine angegebene Zeit warten
		\item /bin/sleep
		\item sleep 0.1h
		\item fstatfs()
	\end{enumerate}
\item wget:
	\begin{enumerate}
		\item Programm um Dateien aus dem Terminal von ftp- oder http-Servern zu laden
		\item /usr/local/bin/wget
		\item wget http://example.com/folder/file
		\item read(), write()
	\end{enumerate}
\end{enumerate}


\section{Aufgabe}
lstat()?
int lstat(const char *path, struct stat *sb);
\begin{lstlisting}
#include <sys/types.h>
#include <sys/stat.h>
#include <stdio.h>
#include <stdlib.h>

int main (int argc, char *argv[]) {
        struct stat st; 

        if (argc != 2) {
                printf(stderr, "Usage: %s <pathname>\n", argv[0]);
                exit(EXIT_FAILURE);
        }   

        if (stat(argv[1], &st) == -1) {
                perror("stat");
                exit(EXIT_FAILURE);
        }   

        printf("File type:                ");

        switch (st.st_mode & S_IFMT) {
                case S_IFBLK:  printf("block device\n");            break;
                case S_IFCHR:  printf("character device\n");        break;
                case S_IFDIR:  printf("directory\n");               break;
                case S_IFIFO:  printf("FIFO/pipe\n");               break;
                case S_IFLNK:  printf("symlink\n");                 break;
                case S_IFREG:  printf("regular file\n");            break;
                case S_IFSOCK: printf("socket\n");                  break;
                default:       printf("unknown?\n");                break;
        }   

        printf("I-node number:            %ld\n", (long) st.st_ino);

        printf("Mode:                     %lo (octal)\n", (unsigned long) st.st_mode);

        printf("Link count:               %ld\n", (long) st.st_nlink);
        printf("Ownership:                UID=%ld   GID=%ld\n", (long) st.st_uid, (long) st.st_gid);

        printf("Preferred I/O block size: %ld bytes\n", (long) st.st_blksize);
        printf("File size:                %lld bytes\n", (long long) st.st_size);
        printf("Blocks allocated:         %lld\n", (long long) st.st_blocks);

        printf("Last status change:       %s", ctime(&st.st_ctime));
        printf("Last file access:         %s", ctime(&st.st_atime));
        printf("Last file modification:   %s", ctime(&st.st_mtime));

        exit(EXIT_SUCCESS);
}

\end{lstlisting}

\end{document}

