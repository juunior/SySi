% Systemmagement und Sicherheit
% uebung2

\documentclass[11pt,a4paper,ngerman]{article}
\usepackage[utf8]{inputenc}
\usepackage[left=2.5cm,right=2.5cm,top=1.5cm,bottom=0.5cm,includeheadfoot]{geometry}
\usepackage{fancyhdr}
\usepackage{titlesec}
\pagestyle{fancy}
\usepackage[makeroom]{cancel}
\fancyhead[L]{C.Wolf, K.Schultz} %Kopfzeile links
\fancyhead[C]{Systemmanagement und Sicherheit} %zentrierte Kopfzeile
\fancyhead[R]{\thepage} %Kopfzeile rechts
\cfoot{}
\fancyfoot[R]{}
\date{06.05.2017}
\author{K. Schultz, C. Wolf}
\title{Systemmanagement und Sicherheit Uebung2}

\titleformat{\section}
{\normalfont\Large\bfseries}{\thesection}{1em.}{}

\usepackage{listings}
\usepackage{color}

\definecolor{dkgreen}{rgb}{0,0.6,0}
\definecolor{gray}{rgb}{0.5,0.5,0.5}
\definecolor{mauve}{rgb}{0.58,0,0.82}

\lstset{
  frame=tb,
  language=C,
  aboveskip=3mm,
  belowskip=3mm,
  showstringspaces=false,
  columns=flexible,
  basicstyle={\small\ttfamily},
  numbers=none,
  numberstyle=\tiny\color{gray},
  keywordstyle=\color{blue},
  commentstyle=\color{dkgreen},
  stringstyle=\color{mauve},
  breaklines=true,
  breakatwhitespace=true,
  tabsize=3,
  escapeinside={(*}{*)}
}

\begin{document}


\section{Aufgabe}
\begin{enumerate}
\item rm:
	\begin{enumerate}
		\item löscht Dateien, aber Standardmäßig keine Directories
		\item /bin/rm
		\item rm test.txt
		\item unlink()
	\end{enumerate}
\item mv:
	\begin{enumerate}
		\item verschiebt Datein, kann auch zum umbennen genutzt werden
		\item /bin/mv
		\item mv test.txt ~/Documents/test.txt
		\item access() und rename()
	\end{enumerate}
\item chmod:
	\begin{enumerate}
		\item \"andert die Zugriffsrecht von Dateien
		\item /bin/chmod
		\item symbolisch: chmod +rwx file, binaer: chmod 777 file
		\item pathconf()
	\end{enumerate}
\item chown:
	\begin{enumerate}
		\item \"andert den Besitzer einer Datei
		\item Debian: /bin/chown, FreeBSD: /usr/bin/chown
		\item chown [user] file
		\item munmap(), exit()
	\end{enumerate}
\item mkdir:
	\begin{enumerate}
		\item erstellt Ordner
		\item /bin/mkdir
		\item mkdir name
		\item sigprocmask()
	\end{enumerate}
\item rmdir:
	\begin{enumerate}
		\item l\"oscht Ordner
		\item /bin/rmdir
		\item rmdir name
		\item execve(), mmap()
	\end{enumerate}
\item kill:
	\begin{enumerate}
		\item sendet \"uber die PID Signale an Prozesse
		\item /bin/kill
		\item kill [PID]
		\item lstat()
	\end{enumerate}
\pagebreak
\item ln:
	\begin{enumerate}
		\item steht fuer link, erzeugt Verkn\"upfung zu Datei oder Ordner
		\item /bin/ln
		\item ln {[option]} ZIEL {[NameDerVerkn\"upfung]}
		\item openat(), fstat()
	\end{enumerate}
\item sleep:
	\begin{enumerate}
		\item laesst den Prozess fuer eine angegebene Zeit warten
		\item /bin/sleep
		\item sleep 0.1h
		\item fstatfs()
	\end{enumerate}
\item wget:
	\begin{enumerate}
		\item Programm um Dateien aus dem Terminal von ftp- oder http-Servern zu laden
		\item /usr/local/bin/wget
		\item wget http://example.com/folder/file
		\item read(), write()
	\end{enumerate}
\end{enumerate}


\section{Aufgabe}

\begin{lstlisting}
#include <sys/types.h>
#include <sys/stat.h>
#include <stdio.h>
#include <stdlib.h>
#include <unistd.h>
#include <string.h>
#include <time.h>
#include <sys/param.h>

int main (int argc, char *argv[]) {
        struct stat st;

        if (argc < 2) {
                printf("Usage: %s <pathname>\n", argv[0]);
                exit(EXIT_FAILURE);
        }

        if (stat(argv[1], &st) == -1) {
                perror(argv[1]);
                exit(EXIT_FAILURE);
        }

        for(int i=1; i < argc; i++){

                lstat(argv[i],&st);
                char type[20];

                if(i > 1){
                        printf("\n\n");
                }                                                                                     
                (*\pagebreak*)                              
                if (S_ISREG(st.st_mode)){                                                             
                        strcpy(type, "regular file");                                                 
                }else if (S_ISLNK(st.st_mode)){                                                       
                        strcpy(type, "symbolic link");                                                
                }else if (S_ISDIR(st.st_mode)){
                        strcpy(type, "directory");
                }else if (S_ISCHR(st.st_mode)){
                        strcpy(type, "character device");
                }else if (S_ISFIFO(st.st_mode)){
                        strcpy(type, "FIFO (named Pipe)");
                }else if (S_ISSOCK(st.st_mode)){
                        strcpy(type,"socket");
                }

                printf("File: \t\t\t %s \n", argv[i]);
                printf("Filetype\t\t %s \n", type);
                printf("UserID:\t\t\t %d \n", st.st_uid);
                printf("GroupID: \t\t %d \n", st.st_gid);
                printf("letzter Zugriff:\t %s", ctime(&st.st_atime));
                printf("letzte Inodeaenderung:\t %s", ctime(&st.st_ctime));
                printf("letzte Aenderung: \t %s", ctime(&st.st_mtime));

                #ifdef __FreeBSD__
                printf("Datei angelegt: \t %s", ctime(&st.st_birthtime));
                #endif

        }


        exit(EXIT_SUCCESS);
}


\end{lstlisting}

\end{document}

